\section{The Problem}
Whilst discussing possible project topics, the team considered various examples of poor interface design  that had impacted their daily lives. Having considered several possible topics, it was decided to focus on self-service ticketing machines, in particular those vending train tickets.
 
Ticket vending machines (TVMs) are becoming increasingly prevalent as an alternative to manned kiosks in an effort to reduce staffing costs and queueing times at railway stations. However, studies have determined that the majority of customers opted to use a manned kiosk instead of a ticketing machine if possible. (PassengerFocus 2008). 

%\begin{quote}
%"Evidence showed that in many cases passengers found operation of the TVMs was not easy and often counter-intuitive (TransportFocus 2010): 
%	\begin{itemize}
%		\item Sheer volume of information felt to be overwhelming and difficult to decide where to press. 
%		\item Passengers felt they had to do the hard work of finding the best ticket for their journey instead of the machine. 
%		\item Information boxes to the side and bottom of screens were often not seen by passengers.  
%		\item Information in yellow writing was not readily visible. 
%		\item In some cases information about routes and restrictions was not provided, instead instructions were given to ask a member of staff for information. 
%		\item Ages applicable to child fares were not shown.  
%		\item Jargon was also used especially in relation to the London zone system."
%	\end{itemize}
%\end{quote} 



An initial study was performed in 2008 with a follow up in 2010,  clearly identify several aspects of the user interfaces which deterred customers from using TVMs, primarily concerning the usability of such systems, especially for those with less technical expertise. It was noted in discussion that several team members had on multiple occasions been required to assist other customers in purchasing their tickets and examination of current ticketing machines from several different rail operators suggested that there were still many issues with the usability of these systems.  
 
We feel that there is much that could be done to improve the user experience when buying a train ticket and that there are many improvements that could be applied to current TVM interfaces in order to increase their popularity and reduce the time taken to buy a ticket, thus reducing queuing time. This project will analyze existing designs from a user-centered view, review existing studies in the area and discuss how current human computer interaction theory can be applied in designing a prototype for a TVM interface with an improved user experience.

\subsection{References (To be moved)} 
PassengerFocus (2008) - Buying a ticket at the station: Research on ticket machine 
TransportFocus. (2016) Retailing. [ONLINE] Available at: http://www.transportfocus.org.uk/key-issues/retailing/. [Accessed 20 October 2016]. 
