\section{Introduction}\label{Int}
The primary aim of the project is to design an interface for a train ticket vending machine through which a customer can buy a ticket or collect pre-booked tickets quickly and easily regardless of level of technological expertise. 
The project will investigate the usability of current ticket vending machines at train stations, conduct a review of studies conducted in the area of interface usability and finally propose a new interface based on user-centered design. Principles of effective human-computer interaction will be used to design an interface that improves upon existing designs, both physical and theoretical. 

%\begin{enumerate}
%	\item Only CV loss is considered
%	\item Only Train loss is considered
%	\item Both CV and Train loss are considered
%\end{enumerate}

%\begin{figure}[h]
%	\centering
%	\includegraphics[width=0.8\linewidth]{images/CVLossANDTrainLoss4}
%	\caption{Comparison of the CV and Train loss for polynomial models of order \textbf{n}, where \textbf{n} = 0 to 4.}
%	\label{fig:CVT4}
%\end{figure}

%\begin{table}[h]
%	\centering
%	\caption{Summary of Model Losses}
%	\label{t:ModLoss}
%	\begin{tabular}{rrrr}
%		\hline
%		\textbf{Order} & \textbf{CV Loss} & \textbf{Train Loss} & \textbf{Average Squared Loss} \\ \hline
%		0 & 10.83 & 8.07 & 178.61 \\
%		1 & 2.71 & 1.54 & 9.03 \\
%		2 & 1.57 & 1.01 & 3.33 \\
%		3 & 3.99 & 0.98 & 12.35 \\
%		4 & 1.64 & 0.93 & 3.30
%	\end{tabular}
%\end{table}


%\begin{figure}[h!] 
%	\centering
%	\begin{subfigure}[b]{0.4\textwidth}
%		\includegraphics[width=\textwidth]{modelNotReg3.png}
%		\caption{Polynomial model with order \textbf{n} = 3}
%		\label{fig:modelNoReg0}
%	\end{subfigure}
%	\begin{subfigure}[b]{0.4\textwidth}
%		\includegraphics[width=\textwidth]{modelNotReg4.png}
%		\caption{Polynomial model with order \textbf{n} = 4}
%		\label{fig:modelNoReg1}
%	\end{subfigure}
%	\caption{The original Olympic men's 400m data (blue crosses) with the polynomial models (red) overlaid without data standardisation.}
%	\label{men400-1noSt}
%\end{figure}


